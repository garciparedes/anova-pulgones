% !TEX root = ./document.tex

\documentclass{article}

\usepackage{mystyle}
\usepackage{myvars}

\begin{document}

  \maketitle


  \section{Descripción del conjunto de datos}

    \paragraph{}
    El conjunto de datos sobre el cual se va a realizar el análisis de la varianza se refiere a una serie de mediciones sobre el número de pulgones por planta de trigo. El experimento fue realizado recogiendo $40$ plantas (muestras aleatorias que supondremos independientes) de trigo, durante un periodo de $6$ semanas.

    \paragraph{}
    [TODO ]

  \section{Cuestiones}

    \subsection{?`Es adecuado utilizar un modelo de un factor para ello? Haz un análisis descriptivo de los datos por semanas y valora las hipótesis que se asumen en el modelo.}

      \paragraph{}
      [TODO ]


    \subsection{Realiza el contraste de igualdad de medias y analiza los residuos. ?`Qué conclusiones sacas?}

      \paragraph{}
      [TODO ]


    \subsection{Realiza el test de Levene. ?`Te sorprende el resultado?}

      \paragraph{}
      [TODO ]


    \subsection{Transforma la respuesta mediante log(recuento+1) y repite el apartado 2. ?`Qué cambios observas?}

      \paragraph{}
      [TODO ]


    \subsection{Realiza el test de kruskal-Wallis sobre los datos originales para contrastar la igualdad de medias}

      \paragraph{}
      [TODO ]

  \section{Código fuente}

    \begin{figure}
      \centering
      \begin{minted}{sas}
        data pulgones;
          do semana=1 to 6;
            do repet=1 to 40;
              input recuento @@;
              output;
            end;
          end;
          datalines;
          12 1 6 1 5 7 1 1 2 1 20 0 9 7 0 12 2 0 0 2 8 0 11 2 21 0 3 18 2 2 6 6
          5 1 12 0 3 1 1 18 40 16 32 15 44 41 43 53 67 21 6 31 15 11 21 40 15 50
          17 32 24 7 25 11 64 22 50 27 3 46 45 10 8 27 34 19 86 83 17 36 86 63
          20 68 55 42 24 29 20 27 26 63 40 46 7 15 10 30 46 26 15 42 6 28 7 9 5
          35 6 9 108 38 35 64 21 20 62 25 0 0 29 2 3 0 4 2 6 7 5 4 6 0 0 5 1 3 2
          2 2 5 0 1 1 0 3 1 2 0 3 3 18 7 21 0 0 0 2 3 0 40 5 7 0 0 0 1 1 2 1 0
          25 1 0 0 0 0 0 0 0 5 0 2 0 0 0 2 0 0 0 4 0 0 0 0 2 0 0 0 0 2 1 0 0 1 7
          0 0 0 4 1 5 2 0 0 0 0 0 0 0 0 0 0 0 0 0 0 0 0 0 0 0 0 0 0 0 0 0 0 0 0
        ;
        run;
      \end{minted}
      \caption{\emph{Código SAS:} Lectura del conjunto de datos}
      \label{code:sas_1}
    \end{figure}


    \begin{figure}
      \centering
      \begin{minted}{sas}
        proc print data=pulgones (obs=5) n;
        run;
      \end{minted}
      \caption{\emph{Código SAS:} Vista preliminar del conjunto de datos}
      \label{code:sas_1}
    \end{figure}


\end{document}
