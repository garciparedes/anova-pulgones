% !TEX root = ./document.tex

\documentclass[11pt]{article}

\usepackage{mystyle}
\usepackage{myvars}

%-----------------------------

\begin{document}

  \maketitle

  %-----------------------------
  %	TEXT
  %-----------------------------


  \section{Descripción del conjunto de datos}
  \label{sec:description}

    \paragraph{}
    El conjunto de datos sobre el cual se va a realizar el análisis de la varianza se refiere a una serie de mediciones sobre el número de pulgones por planta de trigo. El experimento fue realizado recogiendo $40$ plantas (muestras aleatorias que supondremos independientes) de trigo, durante un periodo de $6$ semanas.

    \paragraph{}
    Para la realización de este análisis se ha utilizado la plataforma SAS \cite{sas}, en concreto la \emph{University Edition}. En este caso, el conjunto de datos ha sido suministrado en forma de fragmento de código, el cual se incluye en la figura \ref{code:sas_1}. El conjunto de datos sigue una estructura tabular de $240$ filas (referidas a cada observación) y $3$ columnas (referidas a la \texttt{semana}, identificador de \texttt{muestra} en esa semana y \texttt{recuento} de pulgones en dicha observación) tal y como se muestra en la figura \ref{img:pulgones-print}. El código \emph{SAS} utilizado en este caso se muestra en la figura \ref{code:sas_2}.

    \begin{figure}[h]
      \centering
      \includegraphics[width=0.3\textwidth]{pulgones-print}
      \caption{Visión preliminar del conjunto de datos \texttt{pulgones}}
      \label{img:pulgones-print}
    \end{figure}

  \section{Cuestiones}
  \label{sec:questions}

    \paragraph{}
    El objetivo general del estudio es el siguiente: \textbf{\say{Se trata de analizar si existen diferencias en el número de pulgones por planta entre las diferentes semanas}}, para lo cual se proponen una seria de sub-objetivos que se tratarán de responder en las siguientes secciones.

    \subsection{?`Es adecuado utilizar un modelo de un factor para ello? Haz un análisis descriptivo de los datos por semanas y valora las hipótesis que se asumen en el modelo.}
    \label{sec:e1}

      \paragraph{}
      [TODO ]


    \subsection{Realiza el contraste de igualdad de medias y analiza los residuos. ?`Qué conclusiones sacas?}
    \label{sec:e2}

      \paragraph{}
      [TODO ]


    \subsection{Realiza el test de Levene. ?`Te sorprende el resultado?}
    \label{sec:e3}

      \paragraph{}
      [TODO ]


    \subsection{Transforma la respuesta mediante $log(recuento + 1)$ y repite el apartado \ref{sec:e2}. ?`Qué cambios observas?}
    \label{sec:e4}

      \paragraph{}
      [TODO ]


    \subsection{Realiza el test de kruskal-Wallis sobre los datos originales para contrastar la igualdad de medias}
    \label{sec:e5}
      \paragraph{}
      [TODO ]

  \section{Código fuente}
  \label{sec:code}

    \begin{figure}
      \centering
      \begin{minted}[frame=single,framesep=5pt]{sas}
data pulgones;
  do semana=1 to 6;
    do repet=1 to 40;
      input recuento @@;
      output;
    end;
  end;
  datalines;
  12 1 6 1 5 7 1 1 2 1 20 0 9 7 0 12 2 0 0 2 8 0 11 2 21 0 3 18 2 2 6 6
  5 1 12 0 3 1 1 18 40 16 32 15 44 41 43 53 67 21 6 31 15 11 21 40 15 50
  17 32 24 7 25 11 64 22 50 27 3 46 45 10 8 27 34 19 86 83 17 36 86 63
  20 68 55 42 24 29 20 27 26 63 40 46 7 15 10 30 46 26 15 42 6 28 7 9 5
  35 6 9 108 38 35 64 21 20 62 25 0 0 29 2 3 0 4 2 6 7 5 4 6 0 0 5 1 3 2
  2 2 5 0 1 1 0 3 1 2 0 3 3 18 7 21 0 0 0 2 3 0 40 5 7 0 0 0 1 1 2 1 0
  25 1 0 0 0 0 0 0 0 5 0 2 0 0 0 2 0 0 0 4 0 0 0 0 2 0 0 0 0 2 1 0 0 1 7
  0 0 0 4 1 5 2 0 0 0 0 0 0 0 0 0 0 0 0 0 0 0 0 0 0 0 0 0 0 0 0 0 0 0 0
;
run;
      \end{minted}
      \caption{\emph{Código SAS:} Lectura del conjunto de datos}
      \label{code:sas_1}
    \end{figure}


    \begin{figure}
      \centering
      \begin{minted}[frame=single,framesep=5pt]{sas}
proc print data=pulgones (obs=5) n;
run;
      \end{minted}
      \caption{\emph{Código SAS:} Vista preliminar del conjunto de datos}
      \label{code:sas_2}
    \end{figure}

    %-----------------------------
    %	Bibliographic references
    %-----------------------------

    \nocite{rano2017}

    \bibliographystyle{acm}
    \bibliography{bib}

\end{document}
